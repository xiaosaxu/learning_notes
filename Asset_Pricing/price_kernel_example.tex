\documentclass[uplatex,a4paper]{jsarticle}


\usepackage[uplatex]{otf}
\usepackage[dvipdfmx]{graphicx}
\usepackage{tabularx}
\usepackage{float}
\usepackage{geometry}
\usepackage{listings}
\usepackage{amsmath}
\usepackage{amssymb}
\usepackage{mathrsfs}
\usepackage{url}

\title{CAPMと平均分散モデル \\池田ゼミ講義ノート}
\author{羅洪帥(57165029)}
\date{}

\begin{document}

\maketitle
\hrule
\medskip  

本稿ではプライシングカーネルのフォームについて検討する。

\section{記号と公式の準備}
本稿では、確率変数を$\tilde{X}$と表記する。


確率変数の積の期待値は 期待値の積と共分散の和である。
$
E [\widetilde{m_1} \widetilde{R_1}] = Cov(\widetilde{m_1}, \widetilde{R_1}) +  E [\widetilde{m_1} ] E [ \widetilde{R_1}]
$

共分散に関する公式。
\begin{align*}
Cov(\widetilde{x}, \widetilde{y})
& = E[(\widetilde{x} - \mu_x)(\widetilde{y} - \mu_y)]  \\
& = E[\tilde{x} \tilde{y}] - \mu_x \mu_y  \\
& = E[\tilde{x} (\tilde{y} - \mu_y  )]\\
& = E[(\tilde{x} - \mu_x  ) \tilde{y} ]
\end{align*}

各种定价模型,基本都是怎么用一种多项式来逼近

\section{CAPMの導出}

資産の現在の市場価格$V_0$は1年後のCashFlow$\widetilde{X_1}$の期待値となる。$\widetilde{m_1}$で資産のプライシング・カーネルpricing kernel(stochastic discount factor、確率的割引ファクター)とすれば、下記の式になる。
\begin{equation*}
V_0 = E [\widetilde{m_1} \widetilde{X_1}]
\end{equation*}
注意: 確率変数 $ m_{1}$  は全ての金融資産において共通である。

無リスク資産を1単位を投資する場合に$V_0 = 1$、$X_1 = R_f$となる($R_f$はGross Return)。上式に代入する。
\begin{equation*}
1 = E [\widetilde{m_1} R_f] = R_f E [\widetilde{m_1}]
\end{equation*}
すると、$\displaystyle E [\widetilde{m_1}] = \frac{1}{R_f}$だと分かる。

今度はリスク資産の投資を想定する。1単位のリスク資産の投資による収入を
$\displaystyle \widetilde{R_1} = \frac{\widetilde{X_1}}{V_0}$
と表現すれば、
\begin{equation*}
1 = E [\widetilde{m_1} \widetilde{R_1}] = Cov(\widetilde{m_1}, \widetilde{R_1}) +  E [\widetilde{m_1} ] E [ \widetilde{R_1}]
\end{equation*}
となる。
$ E [\widetilde{m_1}] = \frac{1}{R_f}$を代入する。
\begin{equation*}
E [ \widetilde{R_1}] = R_f \left ( 1- Cov(\widetilde{m_1}, \widetilde{R_1})\right ) =
R_f - R_f Cov(\widetilde{m_1}, \widetilde{R_1})
\end{equation*}
注意: $ - R_f Cov(\widetilde{m_1} \widetilde{R_1})$の部分は即ちリスクプレミアムである。

ここでは$\widetilde m_1$個別資産とマーケットリターンの関係を1次関数と仮定する。
\begin{equation*}
\widetilde m_1 = a -b \widetilde R_m   \quad  (b > 0)
\end{equation*}
以下では、個別資産をjで表現する。$\widetilde m_1$と個別資産の共分散は下式である。
\begin{align*}
Cov(\widetilde{m_1}, \widetilde{R_j})
& = Cov(a - b \widetilde{R_m}, \widetilde{R_j}) \\
& = - b  Cov( \widetilde{R_m}, \widetilde{R_j})
\end{align*}

個別資産リターンとマーケットリターンの期待値は、
\begin{align*}
E [ \widetilde{R_j}]
& = R_f - R_f Cov(\widetilde{m_1}, \widetilde{R_j}) \\
E [ \widetilde{R_m}]
& = R_f - R_f Cov(\widetilde{m_1}, \widetilde{R_m})
\end{align*}
である。
超過リターンの比率は
\begin{equation*}
\frac{E [ \widetilde{R_j}] - R_f }{E [ \widetilde{R_m}] - R_f}
= \frac{Cov(\widetilde{m_1}, \widetilde{R_j})}{Cov(\widetilde{m_1}, \widetilde{R_m}) }
= \frac{-b Cov(\widetilde{R_m}, \widetilde{R_j})}{-b Cov(\widetilde{R_m}, \widetilde{R_m}) }
= \frac{Cov(\widetilde{R_m}, \widetilde{R_j})}{Var[\widetilde{R_m}]}
= \beta_j
\end{equation*}
書き換えると、下式となる。
\begin{equation*}
E [ \widetilde{R_j}] - R_f = \beta_j (E [ \widetilde{R_m}] - R_f)
\end{equation*}


\section{2期間の効用}
$W_0$の富を持つ投資家は今期$C_0$を消費して、残りの$W_0 - C_0$をポートフォリオで運用して、
$(W_0 - C_0) \widetilde R_p = \widetilde C_1$分を来期に消費する。
ここでは、今期の効用と来期の効用は無関係と仮定する。
(注意:これはかなり強い仮定である。)
つまり、
$U(C_0, \widetilde C_1) = U(C_0) + E[U(\widetilde C_1)]$と仮定する。
投資家にとっては、2期間での効用最大化を課題と定義。つまり、
$ max U(C_0) + E[U(\widetilde C_1)]$となる。

ただし、
$E[U(\widetilde C_1)] = E[U(\sum_{j=0}^{J} w_j \widetilde R_j)]  \quad
\text{subject to}  \sum_{j=0}^{J} w_j = 1$ となる。
J種類のリスク資産(j=1,2,3, ... J)と0番の安全資産でポートフォリオを構築される。


ラグランジュ(Lagrange)の未定乗数法を使って、目標関数を以下のように書ける。

$
\displaystyle
\mathscr{L} \equiv u(C_0) + E[u(\sum_{j=0}^{J} w_j \widetilde R_j)] + \lambda ( 1 - \sum_{j=0}^{J} w_j )
$


\begin{align*}
    \frac{\partial \mathscr{L} }{\partial C_0} & =  0 \\
    \frac{\partial \mathscr{L} }{\partial w_j} & =  0
\end{align*}


$
\displaystyle
\frac{\partial \mathscr{L} }{\partial C_0}  =
u'(C_0) + E[u'(\widetilde C_1)(-\sum_{j=0}^{J} w_j \widetilde R_j)]
= 0 \\
u'(C_0) = E[u'(\widetilde C_1)(\sum_{j=0}^{J} w_j \widetilde R_j)]
$

Above is eq1

$
\displaystyle
\frac{\partial \mathscr{L} }{\partial w_j}  =
0 + E[u'(\widetilde C_1)(w_0 - C_0)\widetilde R_j] - \lambda
= 0 \\
E[u'(\widetilde C_1)\widetilde R_j] = \frac{\lambda}{w_0 - C_0}
$

Above is eq2

$
\displaystyle
\frac{\partial \mathscr{L} }{\partial \lambda}  =
1 - \sum_{j=0}^{J} w_j
$

Let eq2 times wj, we have
$
\displaystyle
E[u'(\widetilde C_1) \sum_{j=0}^{J} w_j \widetilde R_j] = \frac{\lambda}{w_0 - C_0} \sum_{j=0}^{J}  w_j= \frac{\lambda}{w_0 - C_0}
$

Above is eq4


Rewrite eq1, we have

\begin{align*}
    u'(C_0)
            &= E[u'(\widetilde C_1)(\sum_{j=0}^{J} w_j \widetilde R_j)] \\
            &= \frac{\lambda}{w_0 - C_0} \\
            &= E[u'(\widetilde C_1)\widetilde R_j]
\end{align*}

Then, we have
\begin{align*}
    1 &= E[ \frac{u'(\widetilde C_1)}{u'(\widetilde C_0)}\widetilde R_j] \\
    E[\widetilde m_1 \widetilde R_j] &=1
\end{align*}


\end{document}

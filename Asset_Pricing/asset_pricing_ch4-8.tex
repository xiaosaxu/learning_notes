\documentclass[uplatex,a4paper]{jsarticle}


\usepackage[uplatex]{otf}
\usepackage[dvipdfmx]{graphicx}
\usepackage{tabularx}
\usepackage{float}
\usepackage{geometry}
\usepackage{listings}
\usepackage{amsmath}
\usepackage{amssymb}
\usepackage{mathrsfs}
\usepackage{url}
\usepackage{color}

\title{線形ファクターモデル}
\author{羅洪帥(57165029)}
\date{}

\begin{document}

\maketitle
\hrule
\medskip  %

CAPMではすべての投資家は世の中のすべての資産を所有することを仮定する。
ただし、取引コストや資本政策(外貨制限など?)と税などの取引の摩擦では、投資家は世界中のすべての
資産を所有することに制限を掛ける。
そして、価格つけなくて、取引もできない資産もある。
例えば、人の労働収入。
市場ポートフォリオのリスク以外に、投資家はこのような分散できないリスクを直面する。
均衡状態では、資産のリスクプレミアムは複数のファクターがあることを想像できる。
CAPMが主張する市場ポートフォリオリスクはリスク資産の唯一のリスクの観点は
実証分析研究に証明されていなかった。

それに対して、APTモデルでは個別リスク資産の収益率は複数のファクターと
固有ファクターに影響されると主張する。
このモデルでは具体的なファクターについて言及していない。
APTは相対的なアセットプライシングモデルで、各リスクファクターのリスクプレミアムと
個別資産はそのプレミアムへの感応度(因子負荷)で資産のリスクプレミアムを確定する。
(CAPMでは、 リスクプレミアムは市場の超過リターンだけで、
各資産はこのファクターに関する感応度は$\beta_i$と定義する。)
APTは投資家の選好に仮定を置いてなくて、裁定取引で資産のリスクプレミアムに
価格をつける。

\section{仮定の整理と記号の準備}
APTのメインの仮定は以下である。

個別資産のリタンは有数個のリスクファクターと線形関係で、
資産の数はリスクファクターの数より多いである。

$k$個のリスクファクターと$n$個の資産を存在し、$n>k$である。
$b_{iz}$は$i$番目資産が$z$番目リスクファクターに対する感応度である。
$\tilde{f_{z}}$はリスクファクター$z$の実現値。
$\tilde{\varepsilon_i}$は資産$i$の個別の固有リスクファクターで、
$k$個のリスクファクター$\tilde{f_{1}},...,\tilde{f_{k}}$
及び他の個別資産の固有ファクター$\tilde{\varepsilon_j}$とは独立する。

資産$i$の期待リタンは$a_i$で表記し、資産$i$の収益は線形モデルでは
$$
\tilde{R_i} = a_i + \sum_{z=1}^{k} b_{iz} \tilde{f_{z}} + \tilde{\varepsilon_i}
\eqno{(3.26)}
$$
となる。

この式のなかでは、
$
E[\tilde{\varepsilon_i}] = E[\tilde{f_z}] = E[\tilde{\varepsilon_i} \tilde{f_z}] = 0
$
、且つ、異なる資産の固有ファクター同士は無相関
$
E[\tilde{\varepsilon_i} \tilde{\varepsilon_j}] = 0, (i \ne j)
$。

簡単化のため、リスクファクターの間も無相関と仮定する。
さらに、リスクファクターの数値は分散1である。
相関あるリスクファクターであっても、 線形変換で独立なリスクファクターに置き換えられる。

最後に、
個別資産の特有リスクは有限であると仮定する。
$
E[\tilde{\varepsilon_i}^2] \equiv s_i^2 < S^2
$
、($ S^2 $ is finite number)。

資産$i$とリスクファクター$z$の共分散は
$
Cov(\tilde{R_i},\tilde{f_z}) = Cov(b_{iz} \tilde{f_z}, \tilde{f_z}) = b_{iz} Cov(\tilde{f_z}, \tilde{f_z}) = b_{iz}
$
であることが分かった。

\section{asymptotic arbitrage  裁定取引}

3.2節の例と異なって、今回は資産の個別の固有リスクがあるため、
リスクフリーのヘッジポートフォリオは構築不可である。
ですが、銘柄数多ければ、個別資産の固有リスクを分散して、排除できる。
裁定取引を排除する上でも、 資産の期待リターンへの制約は足りない場合に、
漸近的な裁定(asymptotic arbitrage)の概念を用いて、
資産のリタンは 個別リスクが存在しない結果に漸近することを説明する。


漸近的な裁定の上限は下記のようである。

(1)ポートフォリオの投資額は0。

(2)銘柄数が多くなれば、分散は0に漸近。

(3)ポートフォリオの収益は0より大きい。


漸近的な裁定がなければ、資産$i$の期待リターンは下記の線形式で与えられる。
$$
a_i = \lambda_0 + \sum_{z=1}^{k} b_{iz} \lambda_z + v_i
$$
$\lambda_0$は定数、$\lambda_z$はリスクプレミアム、$v_i$は期待値からの偏差。

偏差の期待値は0である。
偏差は$k$個のリスクファクターと無相関である。
銘柄数が多ければ、偏差の平方和が0に近づく。

回帰方程式を作る。
$$
a = \begin{bmatrix}
    a_1 \\
    \vdots \\
    a_n
    \end{bmatrix}
= \lambda_0 +
\begin{bmatrix}
    b_11, b_12, \hdots, b_1k \\
    \vdots \\
    b_n1, b_n2, \hdots, b_nk
\end{bmatrix}
\begin{bmatrix}
    \lambda_1 \\
    \vdots \\
    \lambda_n
\end{bmatrix}
+
\begin{bmatrix}
    v_1 \\
    \vdots \\
    v_n
\end{bmatrix}
$$

$W_i$の表記:
$$
W_i = \frac{v_i}{\sqrt{n \sum_{i=1}^{n} v_i^2}}
$$

($W_i$は比率?合計は1?)

HeatonとLucasの実証分析では、証券の所有者自身はSmall Capの所有者もしくは経営者、
その人たちの賃金収入はすでに自社の倒産リスクがあるから、
同様なリスクを避けたい。

\end{document}
